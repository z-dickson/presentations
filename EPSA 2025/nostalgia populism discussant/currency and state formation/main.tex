% Options for packages loaded elsewhere
\PassOptionsToPackage{unicode}{hyperref}
\PassOptionsToPackage{hyphens}{url}
\PassOptionsToPackage{dvipsnames,svgnames,x11names}{xcolor}
%
\documentclass[
  letterpaper,
  DIV=11,
  numbers=noendperiod]{scrartcl}

\usepackage{amsmath,amssymb}
\usepackage{iftex}
\ifPDFTeX
  \usepackage[T1]{fontenc}
  \usepackage[utf8]{inputenc}
  \usepackage{textcomp} % provide euro and other symbols
\else % if luatex or xetex
  \usepackage{unicode-math}
  \defaultfontfeatures{Scale=MatchLowercase}
  \defaultfontfeatures[\rmfamily]{Ligatures=TeX,Scale=1}
\fi
\usepackage{lmodern}
\ifPDFTeX\else  
    % xetex/luatex font selection
\fi
% Use upquote if available, for straight quotes in verbatim environments
\IfFileExists{upquote.sty}{\usepackage{upquote}}{}
\IfFileExists{microtype.sty}{% use microtype if available
  \usepackage[]{microtype}
  \UseMicrotypeSet[protrusion]{basicmath} % disable protrusion for tt fonts
}{}
\makeatletter
\@ifundefined{KOMAClassName}{% if non-KOMA class
  \IfFileExists{parskip.sty}{%
    \usepackage{parskip}
  }{% else
    \setlength{\parindent}{0pt}
    \setlength{\parskip}{6pt plus 2pt minus 1pt}}
}{% if KOMA class
  \KOMAoptions{parskip=half}}
\makeatother
\usepackage{xcolor}
\usepackage[top=30mm,left=20mm,heightrounded]{geometry}
\setlength{\emergencystretch}{3em} % prevent overfull lines
\setcounter{secnumdepth}{-\maxdimen} % remove section numbering
% Make \paragraph and \subparagraph free-standing
\ifx\paragraph\undefined\else
  \let\oldparagraph\paragraph
  \renewcommand{\paragraph}[1]{\oldparagraph{#1}\mbox{}}
\fi
\ifx\subparagraph\undefined\else
  \let\oldsubparagraph\subparagraph
  \renewcommand{\subparagraph}[1]{\oldsubparagraph{#1}\mbox{}}
\fi


\providecommand{\tightlist}{%
  \setlength{\itemsep}{0pt}\setlength{\parskip}{0pt}}\usepackage{longtable,booktabs,array}
\usepackage{calc} % for calculating minipage widths
% Correct order of tables after \paragraph or \subparagraph
\usepackage{etoolbox}
\makeatletter
\patchcmd\longtable{\par}{\if@noskipsec\mbox{}\fi\par}{}{}
\makeatother
% Allow footnotes in longtable head/foot
\IfFileExists{footnotehyper.sty}{\usepackage{footnotehyper}}{\usepackage{footnote}}
\makesavenoteenv{longtable}
\usepackage{graphicx}
\makeatletter
\def\maxwidth{\ifdim\Gin@nat@width>\linewidth\linewidth\else\Gin@nat@width\fi}
\def\maxheight{\ifdim\Gin@nat@height>\textheight\textheight\else\Gin@nat@height\fi}
\makeatother
% Scale images if necessary, so that they will not overflow the page
% margins by default, and it is still possible to overwrite the defaults
% using explicit options in \includegraphics[width, height, ...]{}
\setkeys{Gin}{width=\maxwidth,height=\maxheight,keepaspectratio}
% Set default figure placement to htbp
\makeatletter
\def\fps@figure{htbp}
\makeatother

\addtokomafont{disposition}{\rmfamily}
\KOMAoption{captions}{tableheading}
\usepackage{caption}
\usepackage{hyperref}
\usepackage[ocgcolorlinks]{ocgx2}
\usepackage{xcolor}
\hypersetup{colorlinks,linkcolor={blue},citecolor={purple}, colorlinks=true, citecolor={blue}, urlcolor={blue}}
\makeatletter
\@ifpackageloaded{caption}{}{\usepackage{caption}}
\AtBeginDocument{%
\ifdefined\contentsname
  \renewcommand*\contentsname{Table of contents}
\else
  \newcommand\contentsname{Table of contents}
\fi
\ifdefined\listfigurename
  \renewcommand*\listfigurename{List of Figures}
\else
  \newcommand\listfigurename{List of Figures}
\fi
\ifdefined\listtablename
  \renewcommand*\listtablename{List of Tables}
\else
  \newcommand\listtablename{List of Tables}
\fi
\ifdefined\figurename
  \renewcommand*\figurename{Figure}
\else
  \newcommand\figurename{Figure}
\fi
\ifdefined\tablename
  \renewcommand*\tablename{Table}
\else
  \newcommand\tablename{Table}
\fi
}
\@ifpackageloaded{float}{}{\usepackage{float}}
\floatstyle{ruled}
\@ifundefined{c@chapter}{\newfloat{codelisting}{h}{lop}}{\newfloat{codelisting}{h}{lop}[chapter]}
\floatname{codelisting}{Listing}
\newcommand*\listoflistings{\listof{codelisting}{List of Listings}}
\makeatother
\makeatletter
\makeatother
\makeatletter
\@ifpackageloaded{caption}{}{\usepackage{caption}}
\@ifpackageloaded{subcaption}{}{\usepackage{subcaption}}
\makeatother
\ifLuaTeX
  \usepackage{selnolig}  % disable illegal ligatures
\fi
\usepackage{bookmark}

\IfFileExists{xurl.sty}{\usepackage{xurl}}{} % add URL line breaks if available
\urlstyle{same} % disable monospaced font for URLs
\hypersetup{
  pdftitle={Discussion: Currency and State Formation: Risk, Uncertainty and Voters},
  pdfauthor={Discussant: Zach Dickson (LSE)},
  colorlinks=true,
  linkcolor={blue},
  filecolor={Maroon},
  citecolor={Blue},
  urlcolor={Blue},
  pdfcreator={LaTeX via pandoc}}

\title{Discussion: \emph{Currency and State Formation: Risk, Uncertainty
and Voters}}
\author{Discussant: Zach Dickson (LSE)}
\date{}

\begin{document}
\maketitle

\section{Summary}\label{summary}

This paper investigates the influence of currency and monetary factors
on voting behavior in the 2014 Scottish independence referendum. It
utilizes a two-wave nationally representative survey to argue that
economic calculations, particularly those related to currency risk, were
a central driver of constitutional preferences.

\textbf{Key Strengths of the Paper}

The research makes several significant contributions to the
understanding of how currency formation factors into constitutional
voting behavior. Additionally, it provides a nuanced analysis of the
Scottish independence referendum, which is a critical case study in the
broader context of state formation. The paper's strengths include:

\begin{itemize}
\tightlist
\item
  A Clear Research Gap: The paper correctly identifies that voter
  attitudes toward currency in the context of state formation are an
  under-researched area.
\item
  Multi-Faceted Data: The use of a two-wave panel survey conducted
  before and after the referendum is an asset. Combining this with
  knowledge questions, retrospective evaluations, and survey experiments
  provides a solid, multi-method approach to answering the research
  questions.
\item
  A Compelling Central Finding: The core argument---that uncertainty
  over currency was a powerful factor constraining support for
  independence---is well---supported by the evidence presented.
\item
  A detailed and (perhaps overly) lengthy background on the Scottish
  independence referendum, which provides necessary context for the
  analysis.
\end{itemize}

\section{Discussion}\label{discussion}

The paper's argument is compelling, but there are several areas where it
could be strengthened:

\subsection{1. The Scope of ``State
Formation''}\label{the-scope-of-state-formation}

The title and introduction refer broadly to ``state formation'', and the
appendix lists dozens of new states since 1970. However, the analysis is
exclusively focused on Scotland, a post-industrial democracy with a
highly developed financial system. The lessons from Scotland may not be
easily generalizable to the decolonization or post-Soviet contexts that
make up the bulk of modern state formation.

\textbf{Potential Suggestion:} The paper could be strengthened by either
narrowing the title and framing to be more explicitly about referendums
in advanced democracies or by adding a section to the discussion that
thoughtfully considers how these findings might (or might not) apply to
different contexts of state formation.

\subsection{2. Theory Integration}\label{theory-integration}

The paper does a good job of identifying the role of risk and
uncertainty in voter behavior, particularly in the context of currency.
However, the theoretical framework could be more robust. While
references to economic and psychological literature are present, there's
limited integration into a cohesive theoretical framework explaining how
risk, uncertainty, and identity interact.

\textbf{Potential Suggestion:}

Integrate a more explicit theoretical framework that connects the
concepts of risk tolerance, economic beliefs, and identity to vote
choice. This could involve:

\begin{itemize}
\tightlist
\item
  Drawing on existing theories in political psychology or behavioral
  economics to explain how voters process uncertainty and risk.
\item
  Discussing how identity (national, economic, etc.) influences risk
  perception and decision-making in the context of referendums.
\item
  Outlining how (we should expect) these factors interact in the
  specific context of the Scottish independence referendum, perhaps
  through a conceptual model or diagram.
\end{itemize}

\subsection{3. Clarity of the Survey Experiment
Design}\label{clarity-of-the-survey-experiment-design}

The description of the survey experiment is slightly confusing. It lists
seven conditions (a-e, with b and c split) and states that each was
asked with and without the pound, yielding 14 conditions . However,
condition `a' appears to be the simple baseline question rather than a
treatment.

\textbf{Potential Suggestion:} Add a small table to the methodology
section that clearly lays out the control group and the different
treatment groups in the experiment. This would significantly improve
clarity for the reader.

\subsection{4. Justification of Experimental
Values}\label{justification-of-experimental-values}

The survey experiment uses a hypothetical cost or benefit of ``500 per
month''. This is a very large and specific figure. The paper does not
explain why this amount was chosen. Respondents might perceive it as
unrealistic, which could affect how they answer.

\textbf{Potential Suggestion:}

Briefly justify the choice of this £500 figure in the methods section.
If it was chosen to represent a significant but not impossible shock,
this should be stated. Acknowledging this as a potential limitation in
the discussion would also be appropriate.

\subsection{5. Presentation of Results}\label{presentation-of-results}

\textbf{Density and Interpretation of Regression Tables}

Table 1 and Table 2 are very dense, presenting many models side-by-side
with minimal explanation in the text. This makes it difficult for the
reader to understand the purpose of each model and to follow the
analytical story. Furthermore, the tables report binomial logit
coefficients, which are not directly interpretable in terms of
probability.

\textbf{Potential Suggestions:}

\begin{itemize}
\tightlist
\item
  \textbf{Narrate the Tables:} In the text, guide the reader through the
  tables more explicitly. For example: ``Model 1 presents the baseline
  model with only demographic controls. In Model 2, we introduce our
  general risk tolerance measure, which is a significant predictor of
  the `Yes' vote. In Model 3, we add the belief about retaining the
  pound, and we observe that\ldots{}''
\item
  \textbf{Use Marginal Effects:} For key variables, consider presenting
  average marginal effects instead of or in addition to logit
  coefficients. This would allow for more intuitive statements like, ``A
  voter who believed Scotland would keep the pound was X percentage
  points more likely to vote Yes, holding other factors constant.''
\end{itemize}



\end{document}
